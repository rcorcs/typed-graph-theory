%% This is file `elsarticle-template-1-num.tex',
%%
%% Copyright 2009 Elsevier Ltd
%%
%% This file is part of the 'Elsarticle Bundle'.
%% ---------------------------------------------
%%
%% It may be distributed under the conditions of the LaTeX Project Public
%% License, either version 1.2 of this license or (at your option) any
%% later version.  The latest version of this license is in
%%    http://www.latex-project.org/lppl.txt
%% and version 1.2 or later is part of all distributions of LaTeX
%% version 1999/12/01 or later.
%%
%% Template article for Elsevier's document class `elsarticle'
%% with numbered style bibliographic references
%%
%% $Id: elsarticle-template-1-num.tex 149 2009-10-08 05:01:15Z rishi $
%% $URL: http://lenova.river-valley.com/svn/elsbst/trunk/elsarticle-template-1-num.tex $
%%
\documentclass[preprint,12pt]{elsarticle}

%% Use the option review to obtain double line spacing
%% \documentclass[preprint,review,12pt]{elsarticle}

%% Use the options 1p,twocolumn; 3p; 3p,twocolumn; 5p; or 5p,twocolumn
%% for a journal layout:
%% \documentclass[final,1p,times]{elsarticle}
%% \documentclass[final,1p,times,twocolumn]{elsarticle}
%% \documentclass[final,3p,times]{elsarticle}
%% \documentclass[final,3p,times,twocolumn]{elsarticle}
%% \documentclass[final,5p,times]{elsarticle}
%% \documentclass[final,5p,times,twocolumn]{elsarticle}

\usepackage{amsthm}
\usepackage{amsmath,amssymb}
\usepackage{mathrsfs}
\usepackage{mathtools}
%\usepackage{epsfig,amsfonts,amsthm,amssymb,latexsym,amsmath}
\usepackage{centernot}

\usepackage{graphicx}
\usepackage{subfigure}

\usepackage{color}

\usepackage[section]{placeins}
%% The amsthm package provides extended theorem environments
%% \usepackage{amsthm}

%% The lineno packages adds line numbers. Start line numbering with
%% \begin{linenumbers}, end it with \end{linenumbers}. Or switch it on
%% for the whole article with \linenumbers after \end{frontmatter}.

\usepackage{lineno,hyperref}
\modulolinenumbers[5]

%% natbib.sty is loaded by default. However, natbib options can be
%% provided with \biboptions{...} command. Following options are
%% valid:

%%   round  -  round parentheses are used (default)
%%   square -  square brackets are used   [option]
%%   curly  -  curly braces are used      {option}
%%   angle  -  angle brackets are used    <option>
%%   semicolon  -  multiple citations separated by semi-colon
%%   colon  - same as semicolon, an earlier confusion
%%   comma  -  separated by comma
%%   numbers-  selects numerical citations
%%   super  -  numerical citations as superscripts
%%   sort   -  sorts multiple citations according to order in ref. list
%%   sort&compress   -  like sort, but also compresses numerical citations
%%   compress - compresses without sorting
%%
%% \biboptions{comma,round}

% \biboptions{}

\journal{Discrete Mathematics}


\theoremstyle{plain}
\newtheorem{theorem}{Theorem}[section]
\newtheorem{corollary}[theorem]{Corollary}
\newtheorem{lemma}[theorem]{Lemma}
\newtheorem{prop}[theorem]{Proposition}
\newtheorem{definition}[theorem]{Definition}

\newcommand\tyv[2]{#1\!\!:\!\!#2}
\newcommand\subgraph{\subseteq}

\newcommand{\Image}{\mathrm{Im}}

\newcommand{\etal}{et~al.}


\begin{document}

\begin{frontmatter}

%% Title, authors and addresses

\title{Typed Graph Theory\\Extending graphs with type systems}

%% use the tnoteref command within \title for footnotes;
%% use the tnotetext command for the associated footnote;
%% use the fnref command within \author or \address for footnotes;
%% use the fntext command for the associated footnote;
%% use the corref command within \author for corresponding author footnotes;
%% use the cortext command for the associated footnote;
%% use the ead command for the email address,
%% and the form \ead[url] for the home page:
%%
%% \title{Title\tnoteref{label1}}
%% \tnotetext[label1]{}
%% \author{Name\corref{cor1}\fnref{label2}}
%% \ead{email address}
%% \ead[url]{home page}
%% \fntext[label2]{}
%% \cortext[cor1]{}
%% \address{Address\fnref{label3}}
%% \fntext[label3]{}


%% use optional labels to link authors explicitly to addresses:
%% \author[label1,label2]{<author name>}
%% \address[label1]{<address>}
%% \address[label2]{<address>}

%\tnotetext[mytitlenote]{Fully documented templates are available in the elsarticle package on \href{http://www.ctan.org/tex-archive/macros/latex/contrib/elsarticle}{CTAN}.}

%% Group authors per affiliation:
\author{Rodrigo C. O. Rocha}
\address{School of Informatics\\University of Edinburgh\\Scotland, UK\\r.rocha@ed.ac.uk}
%\fntext[myfootnote]{Since 1880.}

\begin{abstract}
In this paper, we propose \textit{typed graph theory},
a generalisation of graph theory by extending
graphs with type systems.
Type theory, as a study of type systems,
was originally developed as a formal system in logics.
In the proposed typed graph theory, every vertex has a type
and operations are restricted to vertices of a certain type.
We revisit core concepts in graph theory, where
new interesting properties emerge due to the proposed
extension with type systems.
\end{abstract}

\begin{keyword}
graph theory\sep type theory\sep type systems\sep typed graphs
\MSC[2010] 05C60\sep 05C15\sep 05C69\sep 05C05
\end{keyword}

\end{frontmatter}

\linenumbers

\section{Introduction}

In this paper, we extend the concept of a graph with type systems,
called \textit{typed graphs}.
In particular, we revisit several well-known graph properties and operations
where new interesting properties emerge due to the proposed
extension with type systems.
The proposed extension to graph theory with type systems,
that we call \textit{typed graph theory}, is a generalisation of
graph theory, and we assure to keep the generalisation
consistent throughout the paper.

Type systems are subject of study in type theory and programming languages.
In a system of type theory, each term has a type and operations are restricted
to terms of a certain type~\cite{russell1908,church1940}.
Type theory was initially proposed by Bertrand Russel~\cite{russell1908},
and further developed by others such as Whitehead and Hilbert,
in order to address paradoxes in formal logics and rewriting systems.
Afterwards, Alonzo Church~\cite{church1940} developed
simply typed $\lambda$-calculus which
is a typed formalism based on $\lambda$-calculus and 
simple type theory.
In Section~\ref{sec:substitution} we define
a substitution rule with a similar concept to the one used by Church~\cite{church1940}.

%This theoretical concept of typed graphs have been implicitly used in
%many different areas\footnote{
%Typed trees have already been explicitly applied in the area of compiler optimizations
%in computer science, where it was initially developed by the author.
%}.
%However, in this paper we formalize and explore the general concept
%of typed graphs in a more abstract and thorough manner.
%We suggest some possible applications of typed graphs
%in order to highlight its theoretical importance
%from the perspective of supporting those possible applications.
%In computer science, for example, in compiler construction
%we have the following implicit instances of typed graphs:
%call graphs are used in different analysis and optimizations,
%where vertices are functions and functions can have different types;
%data dependence graph where vertices are typed data values
%and the types may differ for example due to type casting;
%typed graphs can also be used to model constant folding,
%simplification of constant expressions,
%if we consider vertices of types \textit{constant} and \textit{non-constant}.
%It can also be applied to other areas such as social networks where
%vertices are individuals and types may represent different niches or communities
%with common interests or goals, or perhaps to neural networks,
%for combining networks with different purposes.

\section{Typed graphs}

In this section, we define typed graphs as undirected graphs extended with type systems,
however, these concepts can be easily extended to directed graphs.

\begin{definition}
Let $T$ be a non-empty set of types
and let $L$ be a set of labels\footnote{Usually we can consider that $L$ is the set of
natural number $\mathbf{N}$.}.
We say that $G = (V,E,T)$ is a typed graph if $(V,E)$ is a graph with
vertex-set $V \subseteq L\times T$ and edge-set $E \subseteq V\times V$.
\end{definition}

We will denote vertices $(v_1, \tau_1) \in V$ by $\tyv{v_1}{\tau_1}$ and edges by $\{\tyv{v_1}{\tau_1}, \tyv{v_2}{\tau_2}\}$.
Notice that two distinct vertices $\tyv{v_1}{\tau_1}, \tyv{v_2}{\tau_2} \in V$ have $v_1 \neq v_2$ but their types $\tau_1$ and $\tau_2$ can be the same.
A homogeneous graph $G$ of type $\tau$ is a typed graph $G = (V,E,T)$ where $T = \{\tau\}$ and therefore all vertices in $V$ have type $\tau$.
Similarly, a \textit{singleton} graph of type $\tau$ is a graph defined as $G = (\{\tyv{v}{\tau}\}, \emptyset, \{\tau\})$.
Graphs in classic graph theory are equivalent to homogeneous typed graphs and thus just special cases of typed graphs.
%This type system, described as typed vertices, creates useful
%restrictions to graph operations (e.g. edge contraction) and relations
%(e.g. isomorphism), as we will discuss further in the remaining of this section.

\begin{definition}[Subgraph]
Let $G_1 = (V_1,E_1,T_1)$ be a typed graph.
$G_2 = (V_2,E_2,T_2)$ is a typed subgraph of $G_1$,
denoted by $G_2 \subgraph G_1$,
if and only if $V_2 \subseteq V_1$,
$E_2 \subseteq E_1$ and $T_2 \subseteq T_1$.
\end{definition}

Similar to the definitions of vertex-induced subgraph and
edge-induced subgraph, we also define \textit{type-induced subgraph},
see Definition~\ref{def:typeindsubgraph}.

\begin{definition}[Type-induced subgraph]\label{def:typeindsubgraph}
Let $G = (V,E,T)$ be a typed graph
and $T'\subseteq T$.
Define the type-induced subgraph of $G$,
denoted by $G[T']$, as the typed subgraph $(V',E',T')$
of $G$ such that
$V' = \{ \tyv{v}{\tau}\in V| \tau\in T'\}$
and
$E' = \{\{\tyv{v_1}{\tau_1}, \tyv{v_2}{\tau_2}\}\in E| \tyv{v_1}{\tau_1}, \tyv{v_2}{\tau_2}\in V'\}$.
\end{definition}

\begin{definition}[Isomorphism]
Let $G_1 = (V_1,E_1,T_1)$ and $G_2 = (V_2,E_2,T_2)$ be typed graphs,
with $V_1 \subseteq L_1\times T_1$ and $V_2 \subseteq L_2\times T_2$.
The two typed graphs $G_1$ and $G_2$ are isomorphic, denoted $G_1 \simeq G_2$,
if and only if there are bijections $\phi:L_1 \rightarrow L_2$ and $\psi:T_1 \rightarrow T_2$
such that
$V_2 = \{\tyv{\phi(v_i)}{\psi(\tau_i)}|\tyv{v_i}{\tau_i}\in V_1\}$ and
$\forall \{\tyv{v_i}{\tau_i}, \tyv{v_j}{\tau_j}\}\in E_1,\{\tyv{\phi(v_i)}{\psi(\tau_i)}, \tyv{\phi(v_j)}{\psi(\tau_j)}\}\in E_2$.
\end{definition}

Whilst both vertex and edge deletion are type-independent operations,
vertex and edge contractions are type-dependent operations in
typed graphs. As we formally define below,
vertex contraction is well-defined only for pairs of vertices of the same type
and, similarly, edge contraction is well-defined only for edges with endpoints of the same type.

\begin{definition}[Vertex contraction]
Let $G = (V,E,T)$ be a typed graph
and $\tyv{v_i}{\tau},\tyv{v_j}{\tau}\in V$ are vertices of the same type $\tau$.
Let $\phi$ be a function which maps every vertex in $V\setminus\{\tyv{v_i}{\tau}, \tyv{v_j}{\tau}\}$ to itself, and otherwise,
maps it to a new vertex $\tyv{w}{\tau}$.
The contraction of vertices $\{\tyv{v_i}{\tau}, \tyv{v_j}{\tau}\}$ results in a new typed graph
$G'=(V',E',T)$,
where $V'=(V\setminus\{\tyv{v_i}{\tau}, \tyv{v_j}{\tau}\})\cup\{\tyv{w}{\tau}\}$ and $E'$ is defined with a correspondence to $E$ such that
for every $v\in V$, $v'=\phi(v)\in V'$ is incident to an edge $e'\in E'$ if and only if, the corresponding edge, $e\in E$ is incident to $v$ in $G$.
\end{definition}

\begin{definition}[Edge contraction]
Let $G = (V,E,T)$ be a typed graph
containing an edge $\varepsilon=\{\tyv{v_i}{\tau}, \tyv{v_j}{\tau}\}$,
with vertices of the same type $\tau$.
Let $\phi$ be a function which maps every vertex in $V\setminus\{\tyv{v_i}{\tau}, \tyv{v_j}{\tau}\}$ to itself, and otherwise,
maps it to a new vertex $\tyv{w}{\tau}$.
The contraction of $\varepsilon$ results in a new typed graph
$G'=(V',E',T)$,
where $V'=(V\setminus\{\tyv{v_i}{\tau}, \tyv{v_j}{\tau}\})\cup\{\tyv{w}{\tau}\}$ and $E'$ is defined with a correspondence to $E\setminus\{\varepsilon\}$ such that
for every $v\in V$, $v'=\phi(v)\in V'$ is incident to an edge $e'\in E'$ if and only if, the corresponding edge, $e\in E\setminus\{\varepsilon\}$ is incident to $v$ in $G$.
\end{definition}

\begin{definition}[Reduced normal form]
A typed graph is in the reduced normal form if and only if for any given edge $\{\tyv{v_i}{\tau_i}, \tyv{v_j}{\tau_j}\}\in E$,
we have $\tau_i \neq \tau_j$.
\end{definition}

If a typed graph is not in the reduced normal form, its reduced normal form can be obtained by
repeatedly contracting edges with endpoints of the same type, until no further 
edge contraction is possible.
Corollary~\ref{cor:uniquereducedgraph} shows that the reduced normal form of a typed graph is unique.
Denote by $\Re_G$ the reduced normal form of a typed graph $G$.
Theorem~\ref{theorem:noreduction}, that follows immediately from the definition,
states that if $G$ is in the reduced normal form then $\Re_G\simeq G$
and no further reduction is possible.

\begin{theorem}\label{theorem:noreduction}
If a typed graph is in the reduced normal form,
no reduction by edge contration is possible.
\end{theorem}
\begin{proof}
Follows immediately from the definition that
a graph in the reduced normal form has no
edge with both endpoints of the same type.
\end{proof}

\begin{theorem}\label{theorem:reducedsubgraph}
If $G$ is a typed graph in the reduced normal form
and $H$ is a subgraph of $G$, then
$H$ is in the reduced normal form.
\end{theorem}
\begin{proof}
%Let $G = (V,E,T)$ be a typed graph in the reduced normal form
%and $H = (V',E',T')$ such that $H \subgraph G$.
%By definition, $\nexists \{\tyv{v_i}{\tau_i}, \tyv{v_j}{\tau_j}\} \in E$
%such that $\tau_i=\tau_j$.
%Because $E'\subseteq E$ we know that $H$ is also in the reduced normal form.
Suppose $H$ is not in the reduced normal form, then there is
an edge $\{\tyv{v_i}{\tau_i}, \tyv{v_j}{\tau_j}\}$ in the edge-set of $H$,
with $\tau_i=\tau_j$, which is a contradiction because $\{\tyv{v_i}{\tau_i}, \tyv{v_j}{\tau_j}\}$
is also in the edge-set of $G$, which is in the reduced normal form.
\end{proof}

\begin{definition}[Homogeneous connected components]\label{def:connectedcomponent}
Let $G = (V,E,T)$ be a typed graph.
Define a relation $ \sim $ on $V$ as follows: for all $\tyv{v_i}{\tau_i}, \tyv{v_j}{\tau_j} \in V$,
define $\tyv{v_i}{\tau_i} \sim \tyv{v_j}{\tau_j}$ if and only if $\tau_i = \tau_j$ and there is at least one
homogeneous path from $\tyv{v_i}{\tau_i}$ to $\tyv{v_j}{\tau_j}$,
where all vertices in the path have the same type $\tau_i$.
The relation $\sim$ is an equivalence relation.
The subgraphs of $G$ induced by the equivalence classes of $\sim $ are
called \textit{homogeneous connected components} of $G$.
\end{definition}

\begin{theorem}\label{theorem:ETCCequiv}
Let $G = (V,E,T)$ be a typed graph
and $S$ be the set of homogeneous connected components of $G$.
Let $G' = (V',E',T)$ be the graph consisting of the homogeneous connected components of $G$ as vertices, i.e.
$V' = \{\tyv{H}{\tau}|H\in S \wedge T(H) = \{\tau\} \}$, and
$E'$ is the set of edges for which $\{\tyv{H_1}{\tau_1},\tyv{H_2}{\tau_2}\}\in E'$
implies that $H_1,H_2\in S$, with $H_1\neq H_2$, and $\exists \{\tyv{v_1}{\tau_1}, \tyv{v_2}{\tau_2}\}\in E$
with $\tyv{v_1}{\tau_1}\in V(H_1)$ and $\tyv{v_2}{\tau_2}\in V(H_2)$.
Therefore $G'$ is isomorphic to $\Re_G$.
\end{theorem}
\begin{proof}
Let $H\in S$. We have that the reduced normal form of $H$ is a \textit{singleton} typed graph,
i.e., $\Re_H \simeq (\{\tyv{w}{\tau}\}, \emptyset, \{\tau\})$,
since, by definition, $H$ is a connected graph and
$\forall \tyv{v_1}{\tau_1}, \tyv{v_2}{\tau_2} \in V(H)$, $\tau_1=\tau_2$.
Therefore $\Re_H$ contracts all edges of $H$, resulting in the \textit{singleton} typed graph,
regardless of the order the edges are contracted.
The resulting contracted \textit{singleton} typed graph corresponds to the
vertex $\tyv{H}{\tau}$ in the typed graph $G'$.
However, edges of $G$ with endpoints of different types are not contracted by $\Re_G$
and by definition they also belong to different homogeneous connected components
of $G$, therefore belonging to both $G'$ and $\Re_G$.
\end{proof}

\begin{corollary}[The reduced normal form uniqueness]\label{cor:uniquereducedgraph}
If $G'$ and $G''$ are reduced normal forms of a typed graph $G$,
$G' \simeq G''$.
\end{corollary}
\begin{proof}
Follows immediately from Theorem~\ref{theorem:ETCCequiv} and also from the fact
that the set of equivalence classes of an equivalence relation is uniquely defined.
\end{proof}

\begin{definition}[Reduced form equivalence]
Let $G_1 = (V_1,E_1,T_1)$ and $G_2 = (V_2,E_2,T_2)$ be typed graphs.
If $\Re_{G_1}$ and $\Re_{G_2}$ are isomorphic, then $G_1$ and $G_2$ are said to
be \textit{reduced form equivalent}, denoted by $G_1 \equiv G_2$.
\end{definition}

Notice that the reduced form equivalence satisfies reflexivity, symmetry and transitivity.
We denote by $[G]_\equiv$ the equivalence class
of a graph $G$ regarding the \textit{reduced form equivalence} relation.
Clearly $|[G]_\equiv|$ is infinite, since each vertex in the reduced normal form
could be a result of reducing any connected typed graph with all vertices having the same
type as the reduced vertex.
We provide further analysis of the \textit{reduced form equivalence} class
in Section~\ref{sec:substitution}.

%Notice that the reduced form equivalence is a transitive relation, i.e.,
%if $G_1 \equiv G_2$ and $G_2 \equiv G_3$, then $G_1 \equiv G_3$,
%where $G_1$, $G_2$ and $G_3$ are typed graphs.

\section{Subgraph substitution rule}\label{sec:substitution}

In this section, we define a substitution rule that maps from
one typed graph to another \textit{reduced form equivalent} typed graph.
We first define two predicates that will be used for defining the 
subgraph substitution rule.

Let $G = (V,E,T)$ be a typed graph and $\tyv{v}{\tau}\in V$.
The first predicate, $P_P(S, \tyv{v}{\tau})$, defines a set $S$ that is a valid superset of the complete partitioning of $N(\tyv{v}{\tau})$,
the set of vertices adjacent to $\tyv{v}{\tau}$.
\begin{equation*}
\begin{split}
P_P(S, \tyv{v}{\tau}): &[\forall S'\in S, S'\neq\emptyset \implies 
  \nexists S''\neq S' \in S, S'\cap S''\neq\emptyset]\wedge \\
  &[\forall \tyv{v'}{\tau'}\in N(\tyv{v}{\tau}), \exists S'\in S, \tyv{v'}{\tau'}\in S'].
\end{split}
\end{equation*}
In other words,
$P_P(S, \tyv{v}{\tau})$ is true for a given set $S$
if and only if
$N_P\subseteq S$, such that $N_P$ is a partition of $N(\tyv{v}{\tau})$,
i.e. $N_P$ contains 
a collection of mutually disjoint non-empty sets whose union is $N(\tyv{v}{\tau})$.

The second predicate, $P_C(V',E',\tau)$, is true if and only if $(V',E',\{\tau\})$ is an 
homogeneous connected component.
\[
P_C(V',E',\tau): \forall \tyv{S'}{\tau}\in V', \forall \tyv{S''}{\tau}\in V',
    \tyv{S'}{\tau}\sim\tyv{S''}{\tau}
\]
where $\sim$ is the relation presented in Definition~\ref{def:connectedcomponent}.

Let $\mathcal{G}(\tyv{v}{\tau})$ such that
\[
\mathcal{G}(\tyv{v}{\tau}) = \{(V',E',\{\tau\}|
    V'=\{\tyv{S'}{\tau}|S'\in S\}\wedge P_P(S, \tyv{v}{\tau}) \wedge P_C(V',E',\tau)\}.
\]
$\mathcal{G}(\tyv{v}{\tau})$ defines the set of all typed graphs $G'\in\mathcal{G}(\tyv{v}{\tau})$
such that $\Re_{G'}$ is isomorphic to the \textit{singleton} graph $(\{\tyv{v}{\tau}\},\emptyset,\{\tau\})$,
i.e., $\mathcal{G}(\tyv{v}{\tau})$ is isomorphically equivalent to the set of all connected homogeneous typed graphs with type-set $\{\tau\}$.

\begin{definition}[Vertex substitution]
Let $G = (V,E,T)$ be a typed graph and $\tyv{v}{\tau}\in V$.
Let $H\in\mathcal{G}(\tyv{v}{\tau})$.
\[
\mathcal{S}^{v:\tau}_H|G = (V',E',T)
\]
such that
$V' = (V\setminus\{\tyv{v}{\tau}\})\cup V(H)$
and
$E' = (E\setminus E'')\cup E(H)\cup E'''$
with
$E'' = \{ \{\tyv{v}{\tau},\tyv{v'}{\tau'}\}| \tyv{v'}{\tau'}\in N(\tyv{v}{\tau})\}$
and
$E''' = \{ \{\tyv{S'}{\tau},\tyv{v'}{\tau'}\}|\tyv{v'}{\tau'}\in N(\tyv{v}{\tau}) \wedge\allowbreak\tyv{S'}{\tau}\in V(H)\}$.
$V(H)$ is the vertex-set of $H$ and $E(H)$ is the edge-set of $H$.
\end{definition}

\begin{theorem}\label{theorem:substEquiv}
Let $G = (V,E,T)$ be a typed graph and $\tyv{v}{\tau}\in V$.
For all $H\in\mathcal{G}(\tyv{v}{\tau})$ we have that $\mathcal{S}^{v:\tau}_H|G \equiv G$.
\end{theorem}

Theorem~\ref{theorem:substEquiv} follows directly from the definition of \textit{vertex substitution}.
$\mathcal{S}^{v:\tau}_H|G$ substitutes a vertex $\tyv{v}{\tau}$ by a connected typed graph
that is \textit{reduced form equivalent} to the \textit{singleton} graph $(\{\tyv{v}{\tau}\},\emptyset,\{\tau\})$.

\begin{theorem}\label{theorem:isEquivClass}
Let $G$ be a typed graph and $\Re_G = (V,E,T)$ be its reduced normal form, where $V = \{\tyv{v_1}{\tau_1}, \tyv{v_2}{\tau_2}, \ldots, \tyv{v_n}{\tau_n}\}, n\in \mathbf{N}$.
Let $K$ be the set
\[
K = \{ \mathcal{S}^{{v_1}:{\tau_1}}_{H_1}|\mathcal{S}^{{v_2}:{\tau_2}}_{H_2}|\dots|\mathcal{S}^{{v_n}:{\tau_n}}_{H_n}|\Re_G
             \,\,\,\,\,|\,\,\,   H_i\in\mathcal{G}(\tyv{v_i}{\tau_i}),\tyv{v_i}{\tau_i}\in V\}
\]
where $\mathcal{S}^{{v_1}:{\tau_1}}_{H_1}|\mathcal{S}^{{v_2}:{\tau_2}}_{H_2}|\cdots|\mathcal{S}^{{v_n}:{\tau_n}}_{H_n}|\Re_G$
represents the consecutive application of the substitution rule, i.e.,
$\mathcal{S}^{{v_1}:{\tau_1}}_{H_1}|\left(\mathcal{S}^{{v_2}:{\tau_2}}_{H_2}|\left(\cdots|\left(\mathcal{S}^{{v_n}:{\tau_n}}_{H_n}|\Re_G\right)\cdots\right)\right)$.
Therefore,
\[
(\forall G'\in K, \exists G''\in[G]_\equiv, G'\simeq G'') \wedge
(\forall G'\in[G]_\equiv, \exists G''\in K, G'\simeq G'')
\]
\end{theorem}
\begin{proof}
It follows directly from the definition of the substitution rule and the definition of $\mathcal{G}(\tyv{v_i}{\tau_i}),\tyv{v_i}{\tau_i}\in V$,
since $\mathcal{G}(\tyv{v_i}{\tau_i})$ is isomorphically equivalent to the set of all connected typed graphs with type-set $\{\tau_i\}$.
By substituting all vertices $\tyv{v_i}{\tau_i}$ in the reduced graph of $G$,
by any other $H_i$ of the same type, we can produce typed graphs isomorphic to
any graph in the equivalence class $[G]_\equiv$.
\end{proof}

Proposition~\ref{prop:sizereducedclass}
analyses the size of a restricted subset of this equivalence class,
which is itself finite if and only if the typed graph in reduced normal form is
also finite.

\begin{prop}\label{prop:sizereducedclass}
Let $G = (V,E,T)$ be a typed graph and $\tyv{v}{\tau}\in V$.
Consider a restricted form of the first predicate that accepts
only sets that is itself a complete partitioning of $N(\tyv{v}{\tau})$, i.e.,
$P_P^*(S, \tyv{v}{\tau}): P_P(S, \tyv{v}{\tau}) \wedge [\forall S'\in S, S'\subseteq N(\tyv{v}{\tau})]$.
In other words,
$P_P(S, \tyv{v}{\tau})$ is true for a given set $S$
if and only if
$S$ itself is a partition of $N(\tyv{v}{\tau})$,
i.e. $S$ contains 
a collection of mutually disjoint non-empty sets whose union is $N(\tyv{v}{\tau})$.
Consider also a restricted subset of $\mathcal{G}(\tyv{v}{\tau})$, namely, $\mathcal{G}^*(\tyv{v}{\tau})$, such that
\[
\mathcal{G}^*(\tyv{v}{\tau}) = \{(V',E',\{\tau\}|
    V'=\{\tyv{S'}{\tau}|S'\in S\}\wedge P_P^*(S, \tyv{v}{\tau}) \wedge P_C(V',E',\tau)\}.
\]
Finally, consider $K \subseteq [G]_\equiv$ such that
\[
K = \{ \mathcal{S}^{{v_1}:{\tau_1}}_{H_1}|\mathcal{S}^{{v_2}:{\tau_2}}_{H_2}|\dots|\mathcal{S}^{{v_n}:{\tau_n}}_{H_n}|G
             \,\,\,\,\,|\,\,\,   H_i\in\mathcal{G}^*(\tyv{v_i}{\tau_i}),\tyv{v_i}{\tau_i}\in V\}
\]
Therefore
\[
\prod_{v:\tau\in V} \binom{d(\tyv{v}{\tau})}{2} \leq |K| < \prod_{v:\tau\in V} \binom{d(\tyv{v}{\tau})}{2}2^{\binom{d(v:\tau)}{2}}
\]
\end{prop}
\begin{proof}
Let $d' = d(\tyv{v}{\tau})$.
Notice that $\{S|P_P^*(S, \tyv{v}{\tau})\}$
is the set of all
valid partitions of the set $N(\tyv{v}{\tau})$
into exactly $d'$ partitions, where some of the partitions may receive no
element of $N(\tyv{v}{\tau})$. We can partition $N(\tyv{v}{\tau})$
in a total of $\binom{d'}{2}$ different ways. Thus $|\{S|P_P^*(S, \tyv{v}{\tau})\}| = \binom{d'}{2}$.

For any valid vertex-set $V'=\{\tyv{S'}{\tau}|S'\in S\}$, with $P_P^*(S, \tyv{v}{\tau})$ true,
the set $\{E'| P_C(V',E',\tau)\}$ represents the set of all edge-sets such that the resulting graph is connected.
Therefore, $|\{E'| P_C(V',E',\tau)\}| < 2^{\binom{d'}{2}}$,
which also follows that $\binom{d'}{2} \leq |\mathcal{G}^*(\tyv{v}{\tau})| < \binom{d'}{2}2^{\binom{d'}{2}}$.

If we repeat this process for all vertices in $V$
we can easily conclude that
\[
\prod_{v:\tau\in V} \binom{d(\tyv{v}{\tau})}{2} \leq |K| < \prod_{v:\tau\in V} \binom{d(\tyv{v}{\tau})}{2}2^{\binom{d(v:\tau)}{2}}
\]
\end{proof}

We have defined and studied a vertex substitution rule.
We can also generalise this substitution rule as a subgraph substitution rule.
Let $G' = (V',E',\{\tau\})$, $\tau\in T$, be a connected homogeneous subgraph of type $\tau$ of $G$,
i.e. $G'\subgraph G$ and $G'$ have all vertices of type $\tau$. 
Let $N(G')$ be the set $\{ \tyv{v}{\tau}\in V\setminus{V'} | \{\tyv{v}{\tau},\tyv{v'}{\tau'}\}\in E \wedge \tyv{v'}{\tau'}\in V'\}$.
We first consider another predicate, $P_P(S, G', \tau)$, that defines a set $S$ that is a valid superset of the complete partitioning of $N(G')$.
\begin{equation*}
\begin{split}
P_P(S, G'): &[\forall S'\in S, S'\neq\emptyset \wedge 
  \nexists S''\neq S' \in S, S'\cap S''\neq\emptyset]\wedge \\
  &[\forall \tyv{v'}{\tau'}\in N(G'), \exists S'\in S, \tyv{v'}{\tau'}\in S'].
\end{split}
\end{equation*}

Let $\mathcal{G}(G')$ such that
\[
\mathcal{G}(G') = \{(V',E',\{\tau\})|
    V'=\{\tyv{S'}{\tau}|S'\in S\}\wedge P_P(S, G') \wedge P_C(V',E',\tau)\}.
\]
$\mathcal{G}(G')$ defines the set of all typed graphs $G''\in\mathcal{G}(\tyv{v}{\tau})$
such that $G''$ is \textit{reduced form equivalent} to $G'$.
The set $\mathcal{G}(G')$ also represents the set of all connected typed graphs with type-set $\{\tau\}$.

\begin{definition}[Subgraph substitution]
Let $G = (V,E,T)$ be a typed graph and
$G'$ be a homogeneous connected subgraph of $G$, where all vertices of $G'$ have the same type $\tau\in T$.
Let $H\in\mathcal{G}(G')$.
\[
\mathcal{S}^{G'}_H|G = (V',E',T)
\]
such that
$V' = (V\setminus V(G'))\cup V(H)$
and
$E' = ((E\setminus E(G'))\setminus E'')\cup E(H)\cup E'''$
where
$E'' = \{ \{\tyv{v}{\tau},\tyv{v'}{\tau'}\}\in E | \tyv{v}{\tau}\in V\setminus{V(G')} \wedge \tyv{v'}{\tau'}\in V(G')\}$
and\\
$E''' = \{ \{\tyv{S'}{\tau'},\tyv{v}{\tau}\}|\tyv{v}{\tau}\in N(G') \wedge \tyv{S'}{\tau'}\in V(H)
           \wedge \tyv{v}{\tau}\in S' \}$.
\end{definition}

\begin{theorem}\label{theorem:subgraphSubstEquiv}
Let $G = (V,E,T)$ be a typed graph and
$G'$, be a connected subgraph of $G$, with all vertices of $G'$ having the same type $\tau\in T$.
For all $H\in\mathcal{G}(G')$ we have that $\mathcal{S}^{G'}_H|G \equiv G$.
\end{theorem}

Theorem~\ref{theorem:subgraphSubstEquiv} follows directly from the definition of \textit{subgraph substitution}.
$\mathcal{S}^{G'}_H|G \equiv G$ substitutes a connected subgraph $G'$ with all vertices having the same type $\tyv{v}{\tau}$,
i.e. $G'$
is \textit{reduced form equivalent} to the \textit{singleton} graph $(\{\tyv{v}{\tau}\},\emptyset,\{\tau\})$,
by another connected typed graph that is also \textit{reduced form equivalent} to the \textit{singleton} graph with type
$\tau$.

\section{Vertex colouring}

Vertex colouring is a labelling of the vertices of a
graph given some restrictions.
In this section, we provide two definitions for vertex colouring
in typed graphs: $(i)$ \textit{type-restrictive vertex colouring}
is a labelling of the vertices such that no two adjacent vertices
or no two vertices of different types can have the same colour;
$(ii)$ \textit{type-permissive vertex colouring} 
is a labelling of the vertices such that 
no two adjacent vertices of the same type can have the
same colour.

\begin{definition}[Type-restrictive vertex colouring]
Let $\kappa_R:V\rightarrow \mathbf{N}$ be the
type-restrictive vertex colouring of a typed graph $G = (V,E,T)$.
For $\tyv{v_i}{\tau_i}, \tyv{v_j}{\tau_j}\in V$,
if $\tau_i\neq\tau_j$
or
$\{\tyv{v_i}{\tau_i}, \tyv{v_j}{\tau_j}\}\in E$
then
$\kappa_R(\tyv{v_i}{\tau_i})\neq\kappa_R(\tyv{v_j}{\tau_j})$.
\end{definition}

\begin{definition}[Type-permissive vertex colouring]
Let $\kappa_P:V\rightarrow \mathbf{N}$ be the
type-permissive vertex colouring of a typed graph $G = (V,E,T)$.
For $\tyv{v_i}{\tau_i}, \tyv{v_j}{\tau_j}\in V$,
if $\tau_i=\tau_j$
and
$\{\tyv{v_i}{\tau_i}, \tyv{v_j}{\tau_j}\}\in E$
then
$\kappa_P(\tyv{v_i}{\tau_i})\neq\kappa_P(\tyv{v_j}{\tau_j})$.
\end{definition}

If the type-set has size one, both typed vertex colouring definitions
simplifies to the classic definition of vertex colouring. 
Theorem~\ref{theorem:vertcolour} shows that both typed vertex colouring definitions
are equivalent when the typed graph has type-set of size one.

\begin{theorem}\label{theorem:vertcolour}
Let $G = (V,E,T)$ be a typed graph.
If $|T|=1$, then
the set of type-restrictive vertex colouring equals
the set of type-permissive vertex colouring.
%Formally:
%For all type-restrictive vertex colouring of $G$,
%$\kappa_R:V\rightarrow \mathbf{N}$,
%there is a 
%type-permissive vertex colouring of $G$,
%$\kappa_P:V\rightarrow \mathbf{N}$,
%such that $\forall \tyv{v}{\tau}\in V, \kappa_R(\tyv{v}{\tau})=\kappa_P(\tyv{v}{\tau})$.
\end{theorem}
\begin{proof}
Consider the two predicates
$P_E(\tyv{v_i}{\tau_i}, \tyv{v_j}{\tau_j}): \{\tyv{v_i}{\tau_i}, \tyv{v_j}{\tau_j}\}\in E$
and
$P_T(\tyv{v_i}{\tau_i}, \tyv{v_j}{\tau_j}):(\tau_i=\tau_j$).
Let $\kappa:V\rightarrow \mathbf{N}$.
Define the following two predicates:
\[
P_R(\kappa)\!: \forall \tyv{v}{\tau}\in V, \nexists \tyv{v'}{\tau'}\neq \tyv{v}{\tau}\in V,
\kappa(\tyv{v}{\tau})=\kappa(\tyv{v'}{\tau'})\wedge(\overline{P_T(\tyv{v}{\tau},\tyv{v'}{\tau'})}\vee P_E(\tyv{v}{\tau},\tyv{v'}{\tau'}))
\]
\[
P_P(\kappa)\!: \forall \tyv{v}{\tau}\in V, \nexists \tyv{v'}{\tau'}\neq \tyv{v}{\tau}\in V,
\kappa(\tyv{v}{\tau})=\kappa(\tyv{v'}{\tau'})\wedge P_T(\tyv{v}{\tau},\tyv{v'}{\tau'})\wedge P_E(\tyv{v}{\tau},\tyv{v'}{\tau'})
\]
Notice that $\kappa:V\rightarrow \mathbf{N}$
is a type-restrictive vertex colouring of $G$
if and only if $P_R(\kappa)$ is true
and that
$\kappa:V\rightarrow \mathbf{N}$
is a type-permissive vertex colouring of $G$
if and only if $P_P(\kappa)$ is true.
Since all vertices have the same type,
$P_T(\tyv{v}{\tau},\tyv{v'}{\tau'})$ is
always true for all 
$\tyv{v}{\tau},\tyv{v'}{\tau'}\in V$.
Therefore both predicates simplify to
\[
P_K(\kappa): \forall \tyv{v}{\tau}\in V, \nexists \tyv{v'}{\tau'}\neq \tyv{v}{\tau}\in V,
(\kappa(\tyv{v}{\tau})=\kappa(\tyv{v'}{\tau'}))\wedge P_E(\tyv{v}{\tau},\tyv{v'}{\tau'})
\]
\end{proof}

Notice also that the predicate $P_K(\kappa:V\rightarrow \mathbf{N})$
is equivalent to the classic definition of vertex colouring.

We call
\textit{type-restrictive chromatic number} the
minimum number of colours by a type-restrictive vertex colouring, denoted by
$\chi_R(G)$.
Similarly, 
we call
\textit{type-permissive chromatic number} the
minimum number of colours by a type-permissive vertex colouring, denoted by
$\chi_P(G)$.

\begin{theorem}
Let $G = (V,E,T)$ be a typed graph in the reduced normal form.
The type-restrictive chromatic number of $G$ is $\chi_R(G)=|T|$.
\end{theorem}
\begin{proof}
By definition, we have that $\chi_R(G)\geq|T|$,
since no two vertices of different types can have the same colour.
However,
$\nexists \{\tyv{v_i}{\tau_i},\tyv{v_j}{\tau_j}\}\in E, \tau_i=\tau_j$,
and thus
all vertices of the same type can have the same colour,
since they are all independent.
Therefore, $\chi_R(G)=|T|$.
\end{proof}

\begin{theorem}
Let $G = (V,E,T)$ be a typed graph in the reduced normal form.
The type-permissive chromatic number of $G$ is $\chi_P(G)=1$.
\end{theorem}
\begin{proof}
By definition, $\forall \{\tyv{v_i}{\tau_i},\tyv{v_j}{\tau_j}\}\in E, \tau_i\neq\tau_j$.
Therefore, by definition,
there is a type-permissive vertex colouring
$\kappa_P:V\rightarrow \mathbf{N}$
such that
$\forall \tyv{v_i}{\tau_i},\tyv{v_j}{\tau_j}\in V, \kappa_P(\tyv{v_i}{\tau_i})=\kappa_P(\tyv{v_j}{\tau_j})$,
since $\tau_i\neq\tau_j$.
Hence, $\chi_P(G)=1$.
\end{proof}

\begin{theorem} \label{theorem:R4moreRestrictive}
In a type-restrictive vertex colouring,
whenever two vertices can have the same colour,
they can also have the same colour in a type-permissive vertex colouring,
but not vice versa. Formally: Let $G = (V,E,T)$ be a typed graph. Then
\begin{equation*}
\begin{split}
\forall \kappa_R:V\rightarrow \mathbf{N}, \forall \tyv{v}{\tau},\tyv{v'}{\tau'}\in V,
\kappa_R(\tyv{v}{\tau})=\kappa_R(\tyv{v'}{\tau'}) \\
\implies \exists \kappa_P:V\rightarrow \mathbf{N}, \kappa_P(\tyv{v}{\tau})=\kappa_P(\tyv{v'}{\tau'})
\end{split}
\end{equation*}
and
\begin{equation*}
\begin{split}
\forall \kappa_P:V\rightarrow \mathbf{N}, \forall \tyv{v}{\tau},\tyv{v'}{\tau'}\in V,
\kappa_P(\tyv{v}{\tau})=\kappa_P(\tyv{v'}{\tau'}) \\
\centernot\implies \exists \kappa_R:V\rightarrow \mathbf{N}, \kappa_R(\tyv{v}{\tau})=\kappa_R(\tyv{v'}{\tau'})
\end{split}
\end{equation*}
such that $\kappa_R:V\rightarrow \mathbf{N}$ is a type-restrictive vertex colouring
and $\kappa_P:V\rightarrow \mathbf{N}$ is a type-permissive vertex colouring.
\end{theorem}
\begin{proof}
Consider again the two predicates
$P_E(\tyv{v_i}{\tau_i}, \tyv{v_j}{\tau_j}): \{\tyv{v_i}{\tau_i}, \tyv{v_j}{\tau_j}\}\in E$
and
$P_T(\tyv{v_i}{\tau_i}, \tyv{v_j}{\tau_j}):(\tau_i=\tau_j)$.
Let us prove the first statement.
By definition,
$\kappa_R(\tyv{v}{\tau})=\kappa_R(\tyv{v'}{\tau'})$ implies
that $P_T(\tyv{v}{\tau},\tyv{v'}{\tau'}) \wedge \overline{P_E(\tyv{v}{\tau},\tyv{v'}{\tau'})}$.
Because $\overline{P_E(\tyv{v}{\tau},\tyv{v'}{\tau'})}$ is true,
i.e. $\{\tyv{v}{\tau},\tyv{v'}{\tau'}\}\notin E$,
by definition it follows that $\exists \kappa_R:V\rightarrow \mathbf{N}, \kappa_R(\tyv{v}{\tau})=\kappa_R(\tyv{v'}{\tau'})$.

Now, let us prove the second statement by contradiction.
Suppose
\begin{equation*}
\begin{split}
\forall \kappa_P:V\rightarrow \mathbf{N}, \forall \tyv{v}{\tau},\tyv{v'}{\tau'}\in V,
\kappa_P(\tyv{v}{\tau})=\kappa_P(\tyv{v'}{\tau'}) \\
\implies \exists \kappa_R:V\rightarrow \mathbf{N}, \kappa_R(\tyv{v}{\tau})=\kappa_R(\tyv{v'}{\tau'}).
\end{split}
\end{equation*}
By definition,
$\kappa_P(\tyv{v}{\tau})=\kappa_P(\tyv{v'}{\tau'})$ implies
that $\overline{P_T(\tyv{v}{\tau},\tyv{v'}{\tau'})} \vee \overline{P_E(\tyv{v}{\tau},\tyv{v'}{\tau'})}$.
Therefore, either $\overline{P_T(\tyv{v}{\tau},\tyv{v'}{\tau'})}$ is true, 
or $\overline{P_E(\tyv{v}{\tau},\tyv{v'}{\tau'})}$ is true, or both are true.
However, if $\overline{P_T(\tyv{v}{\tau},\tyv{v'}{\tau'})}$ is true,
then $\tyv{v}{\tau}$ and $\tyv{v'}{\tau'}$ have different types, i.e., 
$\tau\neq\tau'$.
Therefore, by definition, $\nexists \kappa_R:V\rightarrow \mathbf{N}, \kappa_R(\tyv{v}{\tau})=\kappa_R(\tyv{v'}{\tau'})$,
which is a contradiction.
\end{proof}

Theorem~\ref{theorem:R4moreRestrictive} suggests that
type-restrictive vertex colouring is more prohibitive 
for repeating colours than type-permissive vertex colouring.
Corollary~\ref{cor:colouringrelation} follows directly from Theorem~\ref{theorem:R4moreRestrictive},
because we can always find a 
type-permissive vertex colouring that repeats at least as many colours as
the type-restrictive vertex colouring.

\begin{corollary}\label{cor:colouringrelation}
Let $G = (V,E,T)$ be a typed graph.
Thus $\chi_P(G) \leq \chi_R(G)$.
\end{corollary}

\begin{theorem}
Let $G = (V,E,T)$ be a typed graph, with $|T| = k$,
and $S$ be the set of homogeneous connected components of $G$.
For each $\tau_i \in T$, let $H_i = \bigcup \{S'\in S| T(S') = \{\tau_i\}\}$,
i.e. $H_i$ is the union of the homogeneous connected components of $G$
with all vertices having the same type $\tau_i$.
Let $\kappa_1,\kappa_2,\ldots,\kappa_k:V\rightarrow\mathbf{N}$,
such that $\kappa_i:V\rightarrow\mathbf{N}$ is a
type-restrictive vertex colouring for the subgraph $H_i$, for all $i\in[1,k]$.
If $\Image\,\,\kappa_i \cap \Image\,\,\kappa_j = \emptyset$,
then $\bigcup_{i=1}^k \kappa_i$ is a
type-restrictive vertex colouring of $G$.
\end{theorem}
\begin{proof}
Let us prove by contradiction.
Suppose $\kappa_R = \bigcup_{i=1}^k\kappa_i$ is not a
type-restrictive vertex colouring.
Therefore
$\exists \tyv{v}{\tau},\tyv{v'}{\tau'}\in V$
such that
$\kappa_R(\tyv{v}{\tau})=\kappa_R(\tyv{v'}{\tau'}) \wedge (\tau\neq\tau' \vee \{\tyv{v}{\tau},\tyv{v'}{\tau'}\}\in E)$.
Let us analyse
$\kappa_R(\tyv{v}{\tau})=\kappa_R(\tyv{v'}{\tau'}) \wedge \tau\neq\tau'$
and
$\kappa_R(\tyv{v}{\tau})=\kappa_R(\tyv{v'}{\tau'}) \wedge \{\tyv{v}{\tau},\tyv{v'}{\tau'}\}\in E$
in two separate cases and show that they both never occur.

Because $\tau\neq\tau'$ implies that 
$\tyv{v}{\tau}$ and $\tyv{v'}{\tau'}$ belong two different subgraphs,
resulting from the union of homogeneous connected components of the same type,
and therefore 
$\kappa_R(\tyv{v}{\tau}) \neq \kappa_R(\tyv{v'}{\tau'})$
since $\kappa_i$ have disjoint image sets for different types.
Therefore 
$\kappa_R(\tyv{v}{\tau})=\kappa_R(\tyv{v'}{\tau'}) \wedge \tau\neq\tau'$
is not possible.

For the second case, we have shown that $\tau\neq\tau'$ contradicts 
$\kappa_R(\tyv{v}{\tau})=\kappa_R(\tyv{v'}{\tau'})$, then we consider that
$\tau=\tau'$.
However, $\tau=\tau'\wedge \{\tyv{v}{\tau},\tyv{v'}{\tau'}\}\in E$
implies that 
$\tyv{v}{\tau}$ and $\tyv{v'}{\tau'}$ are in the same
subgraph, resulting from the union of homogeneous connected components of the same type,
and are also adjacent vertices.
Which contradicts the assumption that $k_i$ are valid 
type-restrictive vertex colouring for the subgraph $H_i$, for all $i\in[1,k]$.
Therefore,
$\kappa_R(\tyv{v}{\tau})=\kappa_R(\tyv{v'}{\tau'}) \wedge \{\tyv{v}{\tau},\tyv{v'}{\tau'}\}\in E$
is also not possible.

Since both cases are not possible,
we have a contradiction.
\end{proof}

\begin{corollary}
Let $G = (V,E,T)$ be a typed graph, with $|T| = k$,
and $S$ be the set of homogeneous connected components of $G$.
For each $\tau_i \in T$, let $H_i = \bigcup \{S'\in S| T(S') = \{\tau_i\}\}$,
i.e. $H_i$ is the union of the homogeneous connected components of $G$
with all vertices having the same type $\tau_i$.
Let $\kappa_1,\kappa_2,\ldots,\kappa_k:V\rightarrow\mathbf{N}$,
such that $\kappa_i:V\rightarrow\mathbf{N}$ is a minimum
type-restrictive vertex colouring for the subgraph $H_i$, for all $i\in[1,k]$.
If $\Image\,\,\kappa_i \cap \Image\,\,\kappa_j = \emptyset$,
then $\bigcup_{i=1}^k \kappa_i$ is a minimum
type-restrictive vertex colouring of $G$.
\end{corollary}
\begin{proof}
Suppose $\bigcup_{i=1}^k \kappa_i$ is not a minimum
type-restrictive vertex colouring of $G$.
Then there is a minimum
type-restrictive vertex colouring
$k:V\rightarrow\mathbf{N}$,
such that $|\Image \kappa| < |\Image \bigcup_{i=1}^k \kappa_i|$.
By definition,
if we denote by $\Image[\tau] \kappa$ the image of the colouring
function $\kappa$ for the vertices of type $\tau$,
i.e. $\Image[\tau] \kappa = \{\kappa(\tyv{v}{\tau})| \tyv{v}{\tau}\in V[\tau]\}$,
then we know that $\tau_i,\tau_j\in T, \tau_i\neq\tau_j$ implies that
$(\Image[\tau_i] \kappa)\cap(\Image[\tau_j] \kappa)$, because no two vertices
of different types can have the same colour.
Therefore,
\begin{equation*}
\begin{split}
|\Image \kappa| < |\Image \bigcup_{i=1}^k \kappa_i|\\
|\bigcup_{i=1}^k (\Image[\tau_i] \kappa)| < |\Image \bigcup_{i=1}^k \kappa_i|\\
\sum_{i=1}^k | \Image[\tau_i] \kappa| < \sum_{i=1}^k|\Image \kappa_i|\\
\end{split}
\end{equation*}
and thus
$\exists \tau_i\in T$ such that $| \Image[\tau_i] \kappa < \Image \kappa_i$,
which is a contradiction, since
$\kappa_i:V\rightarrow\mathbf{N}$ is a minimum
type-restrictive vertex colouring for the subgraph $H_i$, for all $i\in[1,k]$.
\end{proof}

In this section, we studied two definitions of vertex colouring.
Although these definitions are consistent with the classic
definition of vertex colouring for homogeneous graphs,
there is no reason to not consider similar
definitions with the opposite type-based restriction,
which we call \textit{negative} definitions.
In the remaining of this section,
we study the \textit{negative} definitions of both 
vertex colouring definitions presented above.
These \textit{negative} vertex colouring definitions
do not simplify to the classic definition of vertex
colouring if we consider homogeneous graphs.

\begin{definition}[Negative type-restrictive vertex colouring]
Let $\overline{\kappa}_R:V\rightarrow \mathbf{N}$ be the
negative type-restrictive vertex colouring of a typed graph $G = (V,E,T)$.
For $\tyv{v_i}{\tau_i}, \tyv{v_j}{\tau_j}\in V$,
if $\tau_i=\tau_j$
or
$\{\tyv{v_i}{\tau_i}, \tyv{v_j}{\tau_j}\}\in E$
then
$\overline{\kappa}_R(\tyv{v_i}{\tau_i})\neq\overline{\kappa}_R(\tyv{v_j}{\tau_j})$.
\end{definition}

\begin{definition}[Negative type-permissive vertex colouring]
Let $\overline{\kappa}_P:V\rightarrow \mathbf{N}$ be the
negative type-permissive vertex colouring of a typed graph $G = (V,E,T)$.
For $\tyv{v_i}{\tau_i}, \tyv{v_j}{\tau_j}\in V$,
if $\tau_i\neq\tau_j$
and
$\{\tyv{v_i}{\tau_i}, \tyv{v_j}{\tau_j}\}\in E$
then
$\overline{\kappa}_P(\tyv{v_i}{\tau_i})\neq\overline{\kappa}_P(\tyv{v_j}{\tau_j})$.
\end{definition}

Theorem~\ref{theorem:negvertcolour} shows that both negative vertex colouring definitions
are equivalent when the typed graph has all vertices with an unique type.

\begin{theorem}\label{theorem:negvertcolour}
Let $G = (V,E,T)$ be a typed graph.
If $|T|=|V|$, i.e.,
$\forall \tyv{v}{\tau} \in V, \nexists \tyv{v'}{\tau'} \in V$
such that $\tyv{v}{\tau} \neq \tyv{v'}{\tau'} \wedge \tau=\tau'$,
then
the set of negative type-restrictive vertex colouring equals
the set of negative type-permissive vertex colouring.
%Formally:
%For all type-restrictive vertex colouring of $G$,
%$\kappa_R:V\rightarrow \mathbf{N}$,
%there is a 
%type-permissive vertex colouring of $G$,
%$\kappa_P:V\rightarrow \mathbf{N}$,
%such that $\forall \tyv{v}{\tau}\in V, \kappa_R(\tyv{v}{\tau})=\kappa_P(\tyv{v}{\tau})$.
\end{theorem}
\begin{proof}
Consider the two predicates
$P_E(\tyv{v_i}{\tau_i}, \tyv{v_j}{\tau_j}): \{\tyv{v_i}{\tau_i}, \tyv{v_j}{\tau_j}\}\in E$
and
$P_T(\tyv{v_i}{\tau_i}, \tyv{v_j}{\tau_j}):(\tau_i=\tau_j$).
Let $\overline{\kappa}:V\rightarrow \mathbf{N}$.
Define the following two predicates:
\[
P_{\overline{R}}(\overline{\kappa})\!: \forall \tyv{v}{\tau}\in V, \nexists \tyv{v'}{\tau'}\neq \tyv{v}{\tau}\in V,
\overline{\kappa}(\tyv{v}{\tau})=\overline{\kappa}(\tyv{v'}{\tau'})\wedge({P_T(\tyv{v}{\tau},\tyv{v'}{\tau'})}\vee P_E(\tyv{v}{\tau},\tyv{v'}{\tau'}))
\]
\[
P_{\overline{P}}(\overline{\kappa})\!: \forall \tyv{v}{\tau}\in V, \nexists \tyv{v'}{\tau'}\neq \tyv{v}{\tau}\in V,
\overline{\kappa}(\tyv{v}{\tau})=\overline{\kappa}(\tyv{v'}{\tau'})\wedge \overline{P_T(\tyv{v}{\tau},\tyv{v'}{\tau'})}\wedge P_E(\tyv{v}{\tau},\tyv{v'}{\tau'})
\]
Notice that $\overline{\kappa}:V\rightarrow \mathbf{N}$
is a negative type-restrictive vertex colouring of $G$
if and only if $P_{\overline{R}}(\overline{\kappa})$ is true
and that
$\overline{\kappa}:V\rightarrow \mathbf{N}$
is a negative type-permissive vertex colouring of $G$
if and only if $P_{\overline{P}}(\overline{\kappa})$ is true.
Since all vertices have the unique types,
$\overline{P_T(\tyv{v}{\tau},\tyv{v'}{\tau'})}$ is
always true for all 
$\tyv{v}{\tau},\tyv{v'}{\tau'}\in V$.
Therefore both predicates simplify to
\[
P_K(\overline{\kappa}): \forall \tyv{v}{\tau}\in V, \nexists \tyv{v'}{\tau'}\neq \tyv{v}{\tau}\in V,
(\overline{\kappa}(\tyv{v}{\tau})=\overline{\kappa}(\tyv{v'}{\tau'}))\wedge P_E(\tyv{v}{\tau},\tyv{v'}{\tau'})
\]
\end{proof}

\section{Special subsets of vertices}

In this section we study special subsets of vertices
in the light of typed graph theory.
First we
define, prove individual properties
and also relations between vertex cover
and independent sets.
Second, we analyse dominating sets in
typed graphs.

\subsection{Vertex cover and independent set}

\begin{definition}[Type-restrictive vertex cover]
Let $G = (V,E,T)$ be a typed graph.
$S\subseteq V$ is a type-restrictive vertex cover
if and only if
$\forall \{\tyv{v_i}{\tau_i},\tyv{v_j}{\tau_j}\}\in E$,
$(\tau_i=\tau_j \implies (\tyv{v_i}{\tau_i}\in S \vee \tyv{v_j}{\tau_j}\in S)
\wedge 
(\tau_i\neq\tau_j \implies (\tyv{v_i}{\tau_i}\in S \wedge \tyv{v_j}{\tau_j}\in S)$.
\end{definition}

A type-restrictive vertex cover must contain
at least one vertex of each edge or both if they have different types.

\begin{definition}[Type-permissive vertex cover]
Let $G = (V,E,T)$ be a typed graph.
$S\subseteq V$ is a type-permissive vertex cover
if and only if
$\forall \{\tyv{v_i}{\tau_i},\tyv{v_j}{\tau_j}\}\in E$,
$\tau_i\neq\tau_j \vee (\tyv{v_i}{\tau_i}\in S \vee \tyv{v_j}{\tau_j}\in S)$.
\end{definition}

A type-permissive vertex cover must contain
at least one vertex of each edge with both endpoints of the same type.

\begin{theorem}
Let $G = (V,E,T)$ be a homogeneous typed graph, i.e. $|T|=1$.
Therefore, $S\subseteq V$ is a type-restrictive vertex cover
if and only if $S$ is also a type-permissive vertex cover.
\end{theorem}
\begin{proof}
Since all vertices have the same type,
$\forall \tyv{v_i}{\tau_i},\tyv{v_j}{\tau_j}\in V, \tau_i=\tau_j$,
then
the definitions of both type-restrictive vertex cover
and type-permissive vertex cover can be simplified
for the homogeneous typed graph $G$.
In particular,
$S\subseteq V$ is a type-restrictive vertex cover of $G$
if and only if
$\forall \{\tyv{v_i}{\tau_i},\tyv{v_j}{\tau_j}\}\in E$,
$\tau_i=\tau_j \implies (\tyv{v_i}{\tau_i}\in S \vee \tyv{v_j}{\tau_j}\in S)$,
because
$\tau_i\neq\tau_j \implies (\tyv{v_i}{\tau_i}\in S \wedge \tyv{v_j}{\tau_j}\in S)$
is always true.
Similarly,
$S\subseteq V$ is a type-permissive vertex cover of $G$
if and only if
$\forall \{\tyv{v_i}{\tau_i},\tyv{v_j}{\tau_j}\}\in E$,
$(\tyv{v_i}{\tau_i}\in S \vee \tyv{v_j}{\tau_j}\in S)$,
because
$\tau_i\neq\tau_j$ is always false.
Therefore, both definitions are equivalent for the simplified 
scenario of homogeneous typed graphs.
\end{proof}

\begin{theorem}
Let $G = (V,E,T)$ be a connected typed graph in the reduced normal form.
Therefore, $S\subseteq V$ is a type-restrictive vertex cover
if and only if $S = V$.
\end{theorem}
\begin{proof}
By definition, $\forall \{\tyv{v_i}{\tau_i},\tyv{v_j}{\tau_j}\}\in E, \tau_i\neq\tau_j$
and there is no isolated vertex in $G$.
Therefore, a type-restrictive vertex cover of $G$ must contain both endpoints
of every edge in $G$.
Hence, $V$ is the only type-restrictive vertex cover of $G$.
\end{proof}

\begin{theorem}
Let $G = (V,E,T)$ be a connected typed graph in the reduced normal form.
Therefore, $\emptyset$ is a type-permissive vertex cover of $G$.
\end{theorem}
\begin{proof}
Let us prove by contradiction.
Suppose $\emptyset$ is not a type-permissive vertex cover of $G$.
Then, $\exists \{\tyv{v_i}{\tau_i},\tyv{v_j}{\tau_j}\}\in E$ such that
$\tau_i=\tau_j \wedge \tyv{v_i}{\tau_i}\notin\emptyset \wedge \tyv{v_j}{\tau_j}\notin\emptyset$,
which is a contradiction
because,$\nexists \{\tyv{v_i}{\tau_i},\tyv{v_j}{\tau_j}\}\in E, \tau_i=\tau_j$.
\end{proof}

\begin{definition}[Type-restrictive independent set]
Let $G = (V,E,T)$ be a typed graph.
$S\subseteq V$ is a type-restrictive independent set
if and only if
$\forall \tyv{v_i}{\tau_i},\tyv{v_j}{\tau_j}\in S$,
$\tyv{v_i}{\tau_i}\neq\tyv{v_j}{\tau_j} \implies (\tyv{v_j}{\tau_j}\notin N(\tyv{v_i}{\tau_i}) \wedge \tau_i=\tau_j)$.
\end{definition}

A type-restrictive independent set contains only independent (i.e. non-adjacent) vertices of the same type.

\begin{definition}[Type-permissive independent set]
Let $G = (V,E,T)$ be a typed graph.
$S\subseteq V$ is a type-permissive independent set
if and only if
$\forall \tyv{v_i}{\tau_i},\tyv{v_j}{\tau_j}\in S$,
$\tyv{v_i}{\tau_i}\neq\tyv{v_j}{\tau_j} \implies (\tyv{v_j}{\tau_j}\notin N(\tyv{v_i}{\tau_i}) \vee \tau_i\neq\tau_j)$.
\end{definition}

\begin{theorem}
Let $G = (V,E,T)$ be a homogeneous typed graph, i.e. $|T|=1$.
Therefore, $S\subseteq V$ is a type-restrictive independent set
if and only if $S$ is also a type-permissive independent set.
\end{theorem}
\begin{proof}
Since all vertices have the same type,
the definitions of both type-restrictive independent set
and type-permissive independent set can be simplified
for the homogeneous typed graph $G$.
In particular,
$S\subseteq V$ is a type-restrictive independent set of $G$
if and only if
$\forall \tyv{v_i}{\tau_i},\tyv{v_j}{\tau_j}\in S$,
$\tyv{v_i}{\tau_i}\neq\tyv{v_j}{\tau_j} \implies \tyv{v_j}{\tau_j}\in N(\tyv{v_i}{\tau_i})$.
because
$\tau_i=\tau_j$
is always true.
Similarly,
$S\subseteq V$ is a type-permissive independent set of $G$
if and only if
$\forall \tyv{v_i}{\tau_i},\tyv{v_j}{\tau_j}\in S$,
$\tyv{v_i}{\tau_i}\neq\tyv{v_j}{\tau_j} \implies \tyv{v_j}{\tau_j}\in N(\tyv{v_i}{\tau_i})$.
because
$\tau_i\neq\tau_j$ is always false.
Therefore, both definitions are equivalent for the simplified 
scenario of homogeneous typed graphs.
\end{proof}

\begin{theorem}
Let $G = (V,E,T)$ be a typed graph in the reduced normal form.
Let $\tau \in T$ and
$V[\tau] = \{\tyv{v_i}{\tau_i}\in V | \tau_i=\tau\}$.
Therefore, $V[\tau]$ is a type-restrictive independent set of $G$.
\end{theorem}
\begin{proof}
By definition, $\forall \{\tyv{v_i}{\tau_i},\tyv{v_j}{\tau_j}\}\in E, \tau_i\neq\tau_j$.
Therefore, $\nexists \tyv{v_i}{\tau_i},\tyv{v_j}{\tau_j} \in V$
such that $\{\tyv{v_i}{\tau_i},\tyv{v_j}{\tau_j}\}\in E \wedge \tau_i=\tau_j$.
Hence, $V[\tau]$ is a set of independent vertices of the same type,
i.e. $V[\tau]$ is a type-restrictive independent set of $G$.
\end{proof}

\begin{theorem}
Let $G = (V,E,T)$ be a typed graph in the reduced normal form.
Therefore, $V$ is a type-permissive independent set of $G$.
\end{theorem}
\begin{proof}
By definition, $\forall \{\tyv{v_i}{\tau_i},\tyv{v_j}{\tau_j}\}\in E, \tau_i\neq\tau_j$.
Suppose $V$ is not a type-permissive independent set.
Thus
$\exists \tyv{v_i}{\tau_i},\tyv{v_j}{\tau_j}\in V,
 \{\tyv{v_i}{\tau_i},\tyv{v_j}{\tau_j}\}\in E \wedge \tau_i=\tau_j$,
which is a contradiction.
Hence, $V$ is a type-permissive independent set of $G$.
\end{proof}

For a typed graph $G$, we define $\overline{G}$ as the 
complement graph with the types of vertices preserved.
The following two theorems show relations between typed
independent sets and the complement graph.

\begin{theorem}
Let $G = (V,E,T)$ be a typed graph and $S\subset V$
be a type-restrictive independent set of $G$.
Thus, $\overline{G}[S]$ is a complete homogeneous
typed graph,
i.e.,
every pair of vertices in $\overline{G}[S]$
are adjacent and of the same type.
\end{theorem}
\begin{proof}
Let us prove by contradiction.
Suppose $\overline{G}[S] = (S,E',T')$
is not a complete homogeneous typed graph.
Then,
$\exists \tyv{v_i}{\tau_i},\tyv{v_j}{\tau_j}\in S,
 \{\tyv{v_i}{\tau_i},\tyv{v_j}{\tau_j}\}\notin E' \vee \tau_i\neq\tau_j$.

First,
$\{\tyv{v_i}{\tau_i},\tyv{v_j}{\tau_j}\}\notin E'$
implies that
$\{\tyv{v_i}{\tau_i},\tyv{v_j}{\tau_j}\}\in E$,
which is a contradiction
since 
$\tyv{v_i}{\tau_i},\tyv{v_j}{\tau_j}\in S$
and $S$ contains no adjacent vertices of $G$.

Second,
$\tau_i\neq\tau_j$
is also a contradiction,
again, because 
$\tyv{v_i}{\tau_i},\tyv{v_j}{\tau_j}\in S$
and $S$ contains only vertices of the same type.
\end{proof}


\begin{theorem}
Let $G = (V,E,T)$ be a typed graph and $S\subset V$
be a type-permissive independent set of $G$.
All
type-induced subgraphs of $\overline{G}[S]$ are
complete homogeneous typed graph,
i.e.,
$\forall \tau\in T$,
$(\overline{G}[S])[\tau]$
is a complete homogeneous typed graph.
\end{theorem}
\begin{proof}
Let us prove by contradiction.
Suppose 
$(\overline{G}[S])[\tau] = (V',E',\{\tau\})$,
for any $\tau\in T$,
is not a complete homogeneous typed graph.
Then,
$\exists \tyv{v_i}{\tau},\tyv{v_j}{\tau}\in V',
 \{\tyv{v_i}{\tau},\tyv{v_j}{\tau}\}\notin E'$.
However,
$\{\tyv{v_i}{\tau},\tyv{v_j}{\tau}\}\notin E'$
implies that
$\{\tyv{v_i}{\tau},\tyv{v_j}{\tau}\}\in E$,
which is a contradiction
since 
$\tyv{v_i}{\tau},\tyv{v_j}{\tau}\in S$
and $S$ contains no adjacent vertices of the same type in $G$.
\end{proof}

\begin{theorem}
Let $G = (V,E,T)$ be a typed graph.
If
$S\subseteq V$
is a type-restrictive vertex cover
then
$V\setminus S$
is a type-permissive independent set.
\end{theorem}
\begin{proof}
Let us prove by contradiction.
Suppose $V\setminus S$ is not a 
type-permissive independent set.
Thus,
$\exists \tyv{v_i}{\tau_i},\tyv{v_j}{\tau_j}\in V\setminus S$
%such that the proposition
%$\tyv{v_i}{\tau_i}\neq\tyv{v_j}{\tau_j} \implies
%(\tyv{v_j}{\tau_j}\notin N(\tyv{v_i}{\tau_i}) \vee \tau_i\neq\tau_j)$.
%is false, i.e.,
%$\tyv{v_i}{\tau_i}\neq\tyv{v_j}{\tau_j} \wedge 
%\{\tyv{v_i}{\tau_i},\tyv{v_j}{\tau_j}\}\in E \wedge \tau_i=\tau_j$
%is true
such that
$\tyv{v_i}{\tau_i}\neq\tyv{v_j}{\tau_j} \wedge 
\{\tyv{v_i}{\tau_i},\tyv{v_j}{\tau_j}\}\in E \wedge \tau_i=\tau_j$,
which is a contradiction,
because we assumed that $\tyv{v_i}{\tau_i},\tyv{v_j}{\tau_j}\in V\setminus S$,
but we know that
$\{\tyv{v_i}{\tau_i},\tyv{v_j}{\tau_j}\}\in E \wedge \tau_i=\tau_j$
implies that
$\tyv{v_i}{\tau_i}\in S \vee \tyv{v_j}{\tau_j}\in S$,
since $S$ is a type-restrictive vertex cover.
\end{proof}

\begin{theorem}
Let $G = (V,E,T)$ be a typed graph.
$S\subseteq V$
is a type-permissive independent set
if and only if
$V\setminus S$
is a type-permissive vertex cover.
\end{theorem}
\begin{proof}
We prove in two steps.

$(\Longrightarrow)$
Let us prove by contradiction.
Suppose $V\setminus S$ is not a 
type-permissive vertex cover.
Thus,
$\exists \{\tyv{v_i}{\tau_i},\tyv{v_j}{\tau_j}\}\in E$
%such that the proposition
%$\tyv{v_i}{\tau_i}\neq\tyv{v_j}{\tau_j} \implies
%(\tyv{v_j}{\tau_j}\notin N(\tyv{v_i}{\tau_i}) \vee \tau_i\neq\tau_j)$.
%is false, i.e.,
%$\tyv{v_i}{\tau_i}\neq\tyv{v_j}{\tau_j} \wedge 
%\{\tyv{v_i}{\tau_i},\tyv{v_j}{\tau_j}\}\in E \wedge \tau_i=\tau_j$
%is true
such that
$\tau_i=\tau_j \wedge \tyv{v_i}{\tau_i}\notin V\setminus S \wedge \tyv{v_j}{\tau_j} \notin V\setminus S$,
i.e.,
$\{\tyv{v_i}{\tau_i},\tyv{v_j}{\tau_j}\}\in E \wedge \tau_i=\tau_j \wedge \tyv{v_i}{\tau_i}\in S \wedge \tyv{v_j}{\tau_j} \in S$,
which is a contradiction
since $S$ is a type-permissive independent set.

$(\Longleftarrow)$
Let us prove by contradiction.
Suppose $V\setminus S$ is not a 
type-permissive independent set.
Thus,
$\exists \tyv{v_i}{\tau_i},\tyv{v_j}{\tau_j}\in V\setminus S$
%such that the proposition
%$\tyv{v_i}{\tau_i}\neq\tyv{v_j}{\tau_j} \implies
%(\tyv{v_j}{\tau_j}\notin N(\tyv{v_i}{\tau_i}) \vee \tau_i\neq\tau_j)$.
%is false, i.e.,
%$\tyv{v_i}{\tau_i}\neq\tyv{v_j}{\tau_j} \wedge 
%\{\tyv{v_i}{\tau_i},\tyv{v_j}{\tau_j}\}\in E \wedge \tau_i=\tau_j$
%is true
such that
$\tyv{v_i}{\tau_i}\neq\tyv{v_j}{\tau_j} \wedge \{\tyv{v_i}{\tau_i},\tyv{v_j}{\tau_j}\}\in E
 \wedge \tau_i=\tau_j$,
i.e.,
for $\tyv{v_i}{\tau_i}\neq\tyv{v_j}{\tau_j}$,
$\tyv{v_i}{\tau_i},\tyv{v_j}{\tau_j}\notin S \wedge \{\tyv{v_i}{\tau_i},\tyv{v_j}{\tau_j}\}\in E
 \wedge \tau_i=\tau_j$,
which is a contradiction
since $\{\tyv{v_i}{\tau_i},\tyv{v_j}{\tau_j}\}\in E
 \wedge \tau_i=\tau_j$ implies that $\tyv{v_i}{\tau_i}\in S \vee \tyv{v_j}{\tau_j}\in S$.
\end{proof}

\begin{theorem}
Let $G = (V,E,T)$ be a typed graph.
If
$S\subseteq V$
is a type-restrictive independent set
then
$V\setminus S$
is a type-permissive vertex cover.
\end{theorem}
\begin{proof}
Let us prove by contradiction.
Suppose $V\setminus S$ is not a 
type-permissive vertex cover.
Thus,
$\exists \{\tyv{v_i}{\tau_i},\tyv{v_j}{\tau_j}\}\in E$
such that
$\tau_i=\tau_j \wedge \tyv{v_i}{\tau_i}\notin V\setminus S \wedge \tyv{v_j}{\tau_j} \notin V\setminus S$,
i.e.,
$\{\tyv{v_i}{\tau_i},\tyv{v_j}{\tau_j}\}\in E \wedge \tau_i=\tau_j \wedge \tyv{v_i}{\tau_i}\in S \wedge \tyv{v_j}{\tau_j} \in S$,
which is a contradiction
since $S$ is a type-restrictive independent set,
and therefore $\tyv{v_i}{\tau_i},\tyv{v_j}{\tau_j}\in S \implies \{\tyv{v_i}{\tau_i},\tyv{v_j}{\tau_j}\}\notin E \wedge \tau_i=\tau_j$.
\end{proof}

\subsection{Dominating set}

\begin{definition}[Dominating set]\label{def:dom-set}
Let $G = (V,E,T)$ be a typed graph.
$S\subseteq V$ is a dominating set
if and only if
$\forall \tyv{v_i}{\tau_i}\in V, \tyv{v_i}{\tau_i}\in S \vee
 (
   \exists \tyv{v_j}{\tau_j}\in S, \tyv{v_j}{\tau_j}\in N(\tyv{v_i}{\tau_i})
   \wedge \tau_i=\tau_j
 )
$.
\end{definition}

If $G$ is a homogeneous typed graph
then Definition~\ref{def:dom-set}
of dominating set is equivalent 
to the classic untyped definition
of dominating set.

\begin{theorem}
If $G = (V,E,T)$ is a typed graph in the reduced normal form,
then $V$ is the only dominating set of $G$.
\end{theorem}
\begin{proof}
Because $G$ is in the reduced normal form, by definition,
$\forall \tyv{v_i}{\tau_i}\in V,
\nexists \tyv{v_j}{\tau_j}\in S, \tyv{v_j}{\tau_j}\in N(\tyv{v_i}{\tau_i})
   \wedge \tau_i=\tau_j$.
Therefore, every vertex $\tyv{v_i}{\tau_i}\in V$ must be in the dominating set.
\end{proof}

\begin{theorem}
Let $G$ be a typed graph.
$\forall \tau \in T$, $S_\tau$ is a dominating set of $G[\tau]$
if and only if $\cup_{\tau\in T} S_\tau$ is a dominating set of $G$.
\end{theorem}
\begin{proof}
$(\Longrightarrow)$ Suppose  
$\forall \tau \in T$, $S_\tau$ is a dominating set of $G[\tau]$.
Hence, $\forall \tyv{v_i}{\tau}\in V$ it holds that 
either $\tyv{v_i}{\tau}\in S_{\tau}$ or there is 
$\tyv{v_j}{\tau}\in S$ such that $\tyv{v_j}{\tau}\in N(\tyv{v_i}{\tau})$.
Therefore, $\cup_{\tau\in T} S_\tau$ is a dominating set of $G$.

$(\Longleftarrow)$ Suppose $S$ is a dominating set of $G$.
Let $S_\tau = \{\tyv{v_i}{\tau_i}\in S|\tau_i=\tau\}$.
Thus, for all vertex $\tyv{v_i}{\tau}\in V$ of type $\tau$,
it holds that 
either $\tyv{v_i}{\tau}\in S$, and then $\tyv{v_i}{\tau}\in S_\tau$,
or there is 
$\tyv{v_j}{\tau}\in S$, i.e. $\tyv{v_j}{\tau}\in S_\tau$,
such that $\tyv{v_j}{\tau}\in N(\tyv{v_i}{\tau})$.
Therefore, $S_\tau$ is a dominating set of $G[\tau]$.
\end{proof}

\begin{corollary}
Let $G$ be a typed graph
and $S$ be the minimum dominating set of $G$.
Thus, $|S|$ is greater than or equal to the number of 
homogeneous connected components of $G$.
\end{corollary}

\begin{definition}[Negative dominating set]\label{def:neg-dom-set}
Let $G = (V,E,T)$ be a typed graph.
$S\subseteq V$ is a dominating set
if and only if
$\forall \tyv{v_i}{\tau_i}\in V, \tyv{v_i}{\tau_i}\in S \vee
 (
   \exists \tyv{v_j}{\tau_j}\in S, \tyv{v_j}{\tau_j}\in N(\tyv{v_i}{\tau_i})
   \wedge \tau_i\neq\tau_j
 )
$.
\end{definition}

If $G$ is a typed graph in the reduced normal form
then Definition~\ref{def:neg-dom-set}
of negative dominating set is equivalent 
to the classic untyped definition
of dominating set.

\begin{theorem}
If $G = (V,E,T)$ is a homogeneous typed graph,
then $V$ is the only negative dominating set of $G$.
\end{theorem}
\begin{proof}
Because $G$ is a homogeneous typed graph, by definition,
$\forall \tyv{v_i}{\tau_i}\in V,
\nexists \tyv{v_j}{\tau_j}\in S, \tyv{v_j}{\tau_j}\in N(\tyv{v_i}{\tau_i})
   \wedge \tau_i\neq\tau_j$.
Therefore, every vertex $\tyv{v_i}{\tau_i}\in V$ must be in the dominating set.
\end{proof}

\section{Binary operations}

Let $G_1 = (V_1,E_1,T_1)$ and $G_2 = (V_2,E_2,T_2)$
be typed graphs.

\begin{definition}[Graph union]
$G_1\cup G_2 = (V_1\cup V_2, E_1\cup E_2, T_1\cup T_2)$
\end{definition}

\begin{definition}[Graph intersection]
$G_1\cap G_2 = (V_1\cap V_2, E_1\cap E_2, T_1\cap T_2)$
\end{definition}

\begin{definition}[Join]
If $V_1\cap V_2=\emptyset$ then we define
$G_1\Join G_2 = (V_1\cup V_2, E_1\cup E_2\cup E', T_1\cup T_2)$
with $E'$ connecting vertices of the same type, i.e.,
$E' = \{\{\tyv{v_1}{\tau_1}, \tyv{v_2}{\tau_2}\}|\tyv{v_1}{\tau_1}\in V_1\wedge\tyv{v_2}{\tau_2}\in V_2\wedge \tau_1=\tau_2\}$
\end{definition}

\begin{definition}[Negative join]
If $V_1\cap V_2=\emptyset$ then we define
$G_1\overline{\Join}G_2 = (V_1\cup V_2, E_1\cup E_2\cup E', T_1\cup T_2)$
with $E'$ connecting vertices of different types, i.e.,
$E' = \{\{\tyv{v_1}{\tau_1}, \tyv{v_2}{\tau_2}\}|\tyv{v_1}{\tau_1}\in V_1\wedge\tyv{v_2}{\tau_2}\in V_2\wedge \tau_1\neq\tau_2\}$
\end{definition}

\begin{prop}
Let $G_1$,$G_2$
be typed graphs and
$H = \Re_{G_1}\overline{\Join}\Re_{G_2}$.
$H \simeq \Re_H$.
\end{prop}
\begin{proof}
Let $\Re_{G_1} = (V_1,E_1,T_1)$ and $\Re_{G_2} = (V_2,E_2,T_2)$.
By definition we have
$H = (V_1\cup V_2, E_1\cup E_2\cup E', T_1\cup T_2)$
with
$E' = \{\{\tyv{v_1}{\tau_1}, \tyv{v_2}{\tau_2}\}|\tyv{v_1}{\tau_1}\in V_1\wedge\allowbreak\tyv{v_2}{\tau_2}\in V_2\wedge \tau_1\neq\tau_2\}$.
Since, by definition, $\forall E_i\in\{E_1,E_2,E'\}$,$\nexists \{\tyv{v_i}{\tau_i}, \tyv{v_j}{\tau_j}\}\in E_i$ where $\tau_i=\tau_j$,
therefore $H$ is in the reduced normal form.
\end{proof}

\begin{prop}
Let $G_1 = (V_1,E_1,T)$ and $G_2 = (V_2,E_2,T)$
be typed graphs with the same type-set and
$H = G_1 \Join G_2$.
For all $\tau\in T$, $\Re_H[\tau]$
is a singleton graph of type $\tau$.
\end{prop}
\begin{proof}
By definition, $\forall \tyv{v_1}{\tau_1}\in V_1$
and $\forall \tyv{v_2}{\tau_2}\in V_2$,
if $\tau_1=\tau_2$ then 
$\tyv{v_1}{\tau_1}$ and
$\tyv{v_2}{\tau_2}$ are adjacent in $H$.
Therefore, $H[\tau]$ is a single
homogeneous connected component in $H$.
As proved in Theorem~\ref{theorem:ETCCequiv},
the reduced normal form of a
homogeneous connected component of type $\tau$ is
a singleton graph of type $\tau$.
\end{proof}

\section{Properties of typed trees}

In this section we study some properties of typed trees.
Typed trees are special cases of typed graphs.
A typed graph $G = (V,E,T)$ is a typed tree
if $(V,E)$ is a tree.

\begin{prop}
Let $G = (V,E,T)$ be a perfect typed n-ary tree, i.e. every non-leaf vertex contains exactly $n$ children,
of height $h$ and $|T| = k$.
Considering that the types are randomly distributed amongst the vertices with a uniform probability.
Let $B(r)$ be the set of all binary $r$-tuples $(b_1, b_2, \ldots, b_r)$,
where $b_i\in\{0,1\}$ for all $b_i$.
Therefore, the expected height of $\Re_G$ is
\[
h\left(1-\sum_{(x_0,x_1,\ldots,x_{h-1})\in B(h)} \prod_{i=0}^{h-1} x_i (\frac{1}{k})^{n^{i+1}}\right)
\]
\end{prop}
\begin{proof}
Given any vertex $\tyv{v}{\tau}$, the probability that all its $n$ children have
the same type $\tau$ is $(\frac{1}{k})^n$.

A reduction happens in a level $l$ of the typed tree if and only if all $n^l$ vertices 
in level $l$ have all their respective children with the same type.
Thus, the reduction happens in a level $l$ with probability
\[
P\{R_l\} = \Big(\Big(\frac{1}{k}\Big)^n\Big)^{n^l} = \Big(\frac{1}{k}\Big)^{n^{l+1}}
\]

Considering the function
\[
p(x_0, x_1, \ldots, x_{h-1}) = \prod_{i=0}^{h-1} x_i P\{R_i\}
\]
where $x_i\in\{0,1\}$ for all $x_i$.

Therefore, the expected height of $\Re_G$ is
\[
h\left(1-\sum_{\mathbf{x}\in B(h)} p(\mathbf{x})\right)
\]
where $p(\mathbf{x})=p(x_0, x_1, \ldots, x_{h-1})$ for
$\mathbf{x}=(x_0, x_1, \ldots, x_{h-1})\in B(h)$.
\end{proof}


\begin{prop}\label{prop:sizereducedclass4trees}
Let $G = (V,E,T)$ be a typed tree in the reduced normal form.
Using de definition of $\mathcal{G}^*(\tyv{v}{\tau})$, $\tyv{v}{\tau}\in V$,
presented in Proposition~\ref{prop:sizereducedclass},
consider $K \subseteq [G]_\equiv$ such that
\[
K \subseteq \{ \mathcal{S}^{{v_1}:{\tau_1}}_{H_1}|\mathcal{S}^{{v_2}:{\tau_2}}_{H_2}|\dots|\mathcal{S}^{{v_n}:{\tau_n}}_{H_n}|G
             \,\,\,\,\,|\,\,\,   H_i\in\mathcal{G}^*(\tyv{v_i}{\tau_i}),\tyv{v_i}{\tau_i}\in V\}
\]
where $K$ contains only typed trees.
Therefore
\[
\prod_{v:\tau\in V} \binom{d(\tyv{v}{\tau})}{2} \leq |K| \leq \prod_{v:\tau\in V} \binom{d(\tyv{v}{\tau})}{2}d(v:\tau)^{d(v:\tau)-2}
\]
\end{prop}
\begin{proof}
Let $d' = d(\tyv{v}{\tau})$.
Again, notice that $\{S|P_P^*(S, \tyv{v}{\tau})\}$
is the set of all
valid partitions of the set $N(\tyv{v}{\tau})$
into exactly $d'$ partitions, where some of the partitions may receive no
element of $N(\tyv{v}{\tau})$. We can partition $N(\tyv{v}{\tau})$
in a total of $\binom{d'}{2}$ different ways. Thus $|\{S|P_P^*(S, \tyv{v}{\tau})\}| = \binom{d'}{2}$.

For any valid vertex-set $V'=\{\tyv{S'}{\tau}|S'\in S\}$, with $P_P^*(S, \tyv{v}{\tau})$ true,
consider the subset $E''\subseteq \{E'| P_C(V',E',\tau)\}$ that represents the set of all edge-sets such that the corresponding graph is a tree.
Since $|V'| \leq d'$, by using Cayley's formula~\cite{cayley1889}, 
the number of trees on $d'$ vertices is $d'^{d'-2}$, i.e., $|E''| \leq d'^{d'-2}$.

If we repeat this process for all vertices in $V$, considering only graphs that represent trees,
we conclude that
\[
\prod_{v:\tau\in V} \binom{d(\tyv{v}{\tau})}{2} \leq |K| \leq \prod_{v:\tau\in V} \binom{d(\tyv{v}{\tau})}{2}d(v:\tau)^{d(v:\tau)-2}
\]
\end{proof}

%\begin{prop}
%Let $G = (V,E,T)$ be a connected perfect typed n-ary tree of height $h$ in the reduced normal form,
%i.e. every non-leaf vertex contains exactly $n$ children with no two
%adjacent vertices of the same type.
%$[G]$ be the equivalence class regarding the \textit{reduced form equivalence} relation.
%Let us denote by $[G]*$ a subset of $[G]$ such that 
%\end{prop}
%\begin{proof}
%\end{proof}

%\subsection{Maximizing isomorphic typed sub-trees}

%Let $G_1 = (V_1,E_1,T_1)$ and $G_2 = (V_2,E_2,T_2)$ be typed trees.
%Let $H_1 = (V_1',E_1',T_1')$ be an induced rooted sub-tree of $G_1$, i.e. $V_1'\subset V_1$, $E_1'\subset E_1$ and $T_1'\in T_1$,
%and similarly, $H_2 = (V_2',E_2',T_2')$ be an induced rooted sub-tree of $G_2$.
%In this paper we consider the problem of maximizing the size of $H_1$ and $H_2$,
%typed sub-trees of $G_1$ and $G_2$ in the reduced form, with $H_1$ isomorphic to $H_2$.
%This problem generalises the problem of producing a SLP vectorization with maximum coverage.

%\begin{figure}
%\centering
%\includegraphics{figs/normalize-tree.pdf}
%\label{fig:normalize-tree}
%\caption{Nomalizing trees}
%\end{figure}

%\section{Conclusions}
%
%As future work, we intend to work on typed graphs with multi-typed vertices.

\section*{Acknowledgments}

%\textbf{Acknowledge the partial support by EPSRC}.

This work was supported by the UK Engineering
and Physical Sciences Research Council (EPSRC) under grants
EP/L01503X/1 for the University of Edinburgh, School
of Informatics, Centre for Doctoral Training in Pervasive
Parallelism (\url{http://pervasiveparallelism.inf.ed.ac.uk/}),
and also by the Institute for Computing Systems Architecture (ICSA)
in the School of Informatics at the University of Edinburgh.

\section*{References}
\bibliographystyle{elsarticle-num}
\bibliography{refs}

\end{document}

%%
%% End of file `elsarticle-template-1-num.tex'.
